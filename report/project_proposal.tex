% !TEX TS-program = xelatex
% !BIB TS-program = bibtex
\documentclass[12pt,letterpaper]{article}
\usepackage{style/dsc180reportstyle} % import dsc180reportstyle.sty

%%%%%%%%%%%%%%%%%%%%%%%%%%%%%%%%%%%%%%%%%%%%%%%%%%%%%%%%
%%%% Title and Authors
%%%%%%%%%%%%%%%%%%%%%%%%%%%%%%%%%%%%%%%%%%%%%%%%%%%%%%%%

\title{DSC Quarter 2 Capstone Project Proposal}

\author{Bradley Nathanson \\
  {\tt bnathanson@ucsd.edu} \\\And
  Christopher Lum \\
  {\tt cslum@ucsd.edu} \\\And
  Trey Scheid \\
  {\tt tscheid@ucsd.edu} \\\And
  Tyler Kurpanek \\
  {\tt tkurpane@ucsd.edu} \\\And
  Yu-Xiang Wang \\
  {\tt yuxiangw@ucsd.edu} \\}

\begin{document}
\maketitle





\section{Proposal}
\subsection{Problem Statement}

Telemetry data is important to privatize as it encodes personally identifiable information which could be used to discover sensitive information. This data is collected from various IT devices, from satellites to personal computers. For our project, the telemetry data includes hardware and software performance metrics, monitoring, and errors. 

We will privatize 22 analysis tasks for the Intel telemetry dataset, ensuring a reasonable privacy budget (). We will implement mechanisms that balance data utility and privacy, ensuring sensitive information is protected, and allocate a reasonable privacy budget (), a parameter that governs the trade-off between accuracy and privacy.

One example of a task is to predict CPU failure. This would require a privatized logistic regression model that predicts the probability of a failure from 0-1. The model would analyze data such as CPU temperature, usage patterns, error logs, or other performance indicators. If non-privatized, this model could expose this data, as a malicious individual could do a reconstruction attack, a method to reconstruct the training data by repeatedly querying the model with various synthetic inputs. The attacker could query this model with different sets of CPU-related inputs, and, over time, the attacker could gain information such as the CPU temperature threshold for an error to occur, or whether certain system configurations have a distinct failure pattern.

\subsection{Methods}
Our methodology for privatizing the 22 telemetry analysis tasks will employ multiple privacy mechanisms, such as the exponential mechanism and the Laplace mechanism, with AutoDP serving as our core privacy accounting tool. For each analysis task, we will first evaluate the sensitivity of the computation and determine the optimal privacy mechanism to maintain utility while satisfying privacy requirements. The implementation process requires careful privacy budget allocation across multiple components of each analysis to ensure the total privacy loss remains within acceptable bounds.

The evaluation of each privatized implementation will involve a comprehensive comparison with non-private baselines to document the privacy-utility tradeoff. This includes analyzing performance metrics before and after applying privacy mechanisms, measuring accuracy degradation at various privacy budget levels, and considering computational efficiency challenges specific to telemetry data analysis. AutoDP will help quantify the privacy guarantees and guide the noise calibration process throughout implementation. 

Each privatized task will be thoroughly documented with implementation details, privacy guarantees, and performance metrics. This documentation will include privacy budget allocation strategies, noise mechanism selection rationale, and practical guidelines for future implementations. The goal is to create a comprehensive resource demonstrating how different privacy mechanisms can be effectively applied to various telemetry analysis scenarios while maintaining practical utility and ensuring strong privacy protections.


\subsection{Deliverable}
The privatized analysis tasks will be stored and shared in a public repository, (without release of source data from Intel). This is our primary contribution, to offer tools in a privatized manner. In collaboration with the accessible programs, we will publish a website that will serve to educate our peers on differential privacy. The variety of analysis tasks done in the telemetry domain can be generalized and applied to many types of data; therefore, descriptions of privacy algorithms, their motivations, and limitations can teach practitioners new methods for their own tasks. 

The Intel data as mentioned is not public (due to the customer privacy and proprietary nature). Therefore our data processing, tasks, and report will include only some metrics of performance and data quality (size, distribution, features, etc). For the information we can share, we will compare the performance of the task with that of the non-private baseline. This gives analysts a sense of the utility-privacy tradeoff in each application. 

\subsection{Impact}
By implementing differential privacy across telemetry we will create a significant impact by maintaining data confidentiality. This project will establish novel approaches to common tasks enabling hardware manufacturers to analyze system performance data while preserving strong privacy guarantees. This advances the field by demonstrating how to maintain data utility while protecting sensitive information in real-world applications. 

The research contribution includes documenting privacy-utility trade-offs and establishing guidelines for privacy budget allocation across multiple analysis tasks. Our work will demonstrate practical privacy considerations in telemetry analysis while protecting users’ participation in datasets. The methodologies developed can be adapted by other researchers working with sensitive telemetry data. 

\subsection{Success Criteria}
The success of this project is dependent on a few factors. The first two are team collaboration and schedule adherence. There are many tasks that can be privatized and there may be unique challenges for each (hence the value in sharing these!). With one-quarter complete with group work on our privatized logistic regression paper, our group is confident in our communication, task management, and problem-solving abilities. Paired with our mentor Yu-Xiang Wang, an expert in the field of differential privacy, and a seasoned professor, we are equipped to find innovative and theoretically founded methods for privatizing data tasks.

The other requirements for this project rely on data access and task availability. The Intel data is proprietary, and we have signed agreements to use the data for research, however strict access and usage terms have not been given to us yet. Previous students have worked with the contact/program at Intel successfully and we are reassured by them that we will have a usable telemetry dataset by the start of the quarter.  Similarly, there is a set of non-privatized tasks completed on this dataset by previous data scientists, their work is the foundation which we will build off of to show utility is possible even with privacy. These projects were successful implementations on the specific dataset we will have access to, this pairing therefore will continue to bear fruit as we privatize the tasks and compare baselines. 

Lastly, although we have not reviewed the dataset and tasks yet (no access), the intel program is sharing genuine telemetry information from devices with given consent as part of their program. Additionally, this HDSI-Intel partnership has been cooperating since 2020 and HDSI has used hundreds of terabytes of information. 





\end{document}